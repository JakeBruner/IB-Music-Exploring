% !TeX spellcheck = en_US
\documentclass[11pt,a4paper]{article}
\usepackage[utf8]{inputenc}
\usepackage{amsmath}
\usepackage{amsfonts}
\usepackage{amssymb}
\usepackage{graphicx}
\usepackage{mathtools}
\usepackage[hidelinks]{hyperref}  % most people dont know of this :3

\usepackage{subfig}
% \usepackage[margin=0.7in]{geometry}

% \usepackage[backend=bibtex,style=verbose-ibid]{biblatex}
% \addbibresource{citations.bib}

\usepackage{listings}
\usepackage{color}
\definecolor{dkgreen}{rgb}{0,0.6,0}
\definecolor{gray}{rgb}{0.5,0.5,0.5}
\definecolor{mauve}{rgb}{0.58,0,0.82}

\lstset{frame=tb,
  language=Python,
  aboveskip=3mm,
  belowskip=3mm,
  showstringspaces=false,
  columns=flexible,
  basicstyle={\small\ttfamily},
  numbers=none,
  numberstyle=\tiny\color{gray},
  keywordstyle=\color{blue},
  commentstyle=\color{dkgreen},
  stringstyle=\color{mauve},
  breaklines=true,
  breakatwhitespace=true,
  tabsize=3
}


\newcommand{\inv}{^{\raisebox{.2em}{$\scriptscriptstyle-1$}}}
\newcommand{\qed}{\hfill $\blacksquare$}

\newcommand{\integers}{\mathbb{Z}}
\newcommand{\rationals}{\mathbb{Q}}
\newcommand{\reals}{\mathbb{R}}
\newcommand{\complexes}{\mathbb{C}}
\newcommand{\field}{\mathbb{F}}

\author{Jacob Bruner}
\title{IB Music Exploring Portfolio}
\date{\today}

\begin{document}
\maketitle
\tableofcontents

\pagebreak

\iffalse
############
heres an example of a code block
\begin{lstlisting}
        def intervalValues(z, n):
            return output # return the sequence of values
\end{lstlisting}

heres an example of an image
\begin{figure}[h]
\begin{center}
\includegraphics[scale=.37]{onefifteen} 
\caption{Sequences Generated by n = 1-15 on Argand Diagram}
\end{center}
\end{figure}
############
\fi

\section{Introduction}
In my explorational journey, I frequently encounter new styles, genres, techniques and theory that challenge my assumptions about music. Across a number of AOIs, I find myself pouring through musical scores, in combination with listening, sight-reading, and analyzing, trying to grasp at a genre’s conventions or motifs. One particular genre that piqued my interest was videogame music. To me, it’s fascinating how videogames leverage techniques from a deluge of genres, yet still conform to the technical or physical limitations of a game. For instance, a number of songs I analyzed utilize only the square, sawtooth, and triangle tones due to the limitations of the original NES, making complex instrumentation or intricate polyphony impossible. Despite this, video game composers manufacture memorable, impactful music that often stands alone as a work of art, irrespective of the game it's in. This provoked me to explore how this type of music could act as both music for listening (AOI 2) and music to complement a game or other medium (AOI 3). I researched the intergenre connections among videogames music, classical impressionism and early ragtime/jazz, discovering surprising connections. To experiment with these fascinating ideas, this led me to perform a cover of a videogame classic in the style of ragtime and to create an arrangement of an orchestral videogame song for piano duet, falling under AOI 2 and 3 respectively.




\end{document}